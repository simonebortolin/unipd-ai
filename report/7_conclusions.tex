\section{Conclusion}
In this paper, we have analyzed how evacuation and allocation of people in case of disasters and wars can be dealt using artificial intelligence and, in particular, matching theory.

We have observed that, discarding models that don’t respect ethic principles, for our case of study only two mechanisms are adaptable: School Choice Problem and Machine learning-based Matching. The latter model is an optimization model based on the contribution of machine learning and, since its output value \(\pi\) is very similar to the list of priorities of the School Choice Problem, we can say that in both models COM is not violated.

The main advantage of a matching model that can be adapted in various situations is that it allows you to work and improve a single model that can be reused also for many other purposes. In this way, it’s possible to reduce effort on building more stable matching resolution algorithms.

Finally, we have studied how these models can be very useful if applied in the particular case of a global issue such as war that, unfortunately, is still very current. In fact:
\begin{itemize}
    \item 1\(^{st}\) model can redistribute people in safe countries;
    \item 2\(^{nd}\) model can manage the evacuation from buildings at risk and the shelter to bunkers or other safe places.
\end{itemize}
