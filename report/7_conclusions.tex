\section{Conclusion}
In this paper, we have analyzed how evacuation and allocation of people in case of disasters and wars can be dealt using artificial intelligence, in particular matching theory.

We have seen that, discarding models that don’t respect ethic principles, only two mechanisms are adaptable for our case of study: School Choice Problem and Machine learning-based Matching. The latter model is an optimization model based on the contribution of machine learning and, since its output value \(\pi\) is very similar to the list of priorities of the School Choice Problem, we can say that also in this model COM is not violated.

The main advantage of a matching model that can be adapted in various situations is that it allows you to work and improve a single model that can be also reused for many other purposes. Thanks to this, it’s possible to reduce effort on building increasingly stable matching resolution algorithms.

