\subsubsection{Use of matching models in case of war}\label{use-of-matching-model-in-case-of-war}%
The previous models, as well guessed are very important in case of war: the first model is used just to redistribute in a fair way people in case of mass escapes, the second one to optimize the evacuation
towards bunkers or other safe places in case of attacks and attempts to buildings.
The second model can be widely used to manage the evacuation and moving to safe locations of buildings, hospitals, entire campuses, cities and counties.
The second model, by modifying the limitations of 60 people and 60 metres, can be widely used to manage the evacuation and moving to safe locations (like safe place or bunker) of buildings, hospitals,
entire campuses, cities and counties from dangerous areas for fire, tsunamis, war, earthquake, risk of flooding, and much more.
In~\citet{delacretaz_2020,delacretaz_2019,delacretaz_2016} there are other example cases of use of this models.

\subsubsection{Similarities and differences between mechanisms}\label{similarities-and-differences-between-mechanisms}

As we have seen the two previous models are very similar, there are few differences and it is possible to use a
common system for the analysis and to realize the matching for both cases.
In fact in~\citet{delacretaz_2020} the model has been generalized to be adepted in more situations.
%in the table \ref{tab:summaryview} is a summary of the differences and similarities.
%
%\begin{table}[!hbt]
%    \centering
%    \begin{tabular}{p{0.15\columnwidth}|p{0.35\columnwidth}|p{0.35\columnwidth}}
%        \hline {} & {Model for asylum} & {Model for evacuation} \\
%        \hline Input matrix & \(C \): countries, \(R \): refugees & \(E \): exit, \( P \): people \\
%        \hline Input size & \(|C|= m\), \( |R|= n\), \( n \gg m \) & \(|E|= m\), \( |P|= n\), \( n \ll m \) \\
%        \hline Constrain & \(P_c\): preference of countries, \( P_r\): preference of refugee & \(N_e\): preference of exits, \( N_p\): preference of people  \\
%        \hline
%    \end{tabular}
%    \caption{Summary view of model similarities and differences.}
%    \label{tab:summaryview}
%\end{table}
