\subsection{Use of matching models in case of war}\label{use-of-matching-model-in-case-of-war}%


Previous models argue topics that can be very useful in case of war. In particular, for mass escapes, first model can redistribute in a quick and fair way people to safe countries. Then, during bombardments or alarms of incoming attacks, second model can optimally manage:

\begin{itemize}
    \item evacuation of people from buildings at risk of attack;
    \item shelter of people towards bunkers or other safe places.
\end{itemize}


\subsubsection{Similarities and differences between mechanisms}\label{similarities-and-differences-between-mechanisms}

As described in this paper, the two proposed models, except for few differences, are very similar. For this reason it's possible to use a common system to analyse and solve the two matching problems. ~\citet{delacretaz_2020} formulated a generalized model that can be adapted in many situations, like ours.
