\section{Matching models}\label{matching-models}%


\subsection{Matching model for refugee resettlement}\label{matching-model-for-refugee-resettlement}%

To avoid that refugee families seeking shelter are assigned to countries randomly, \citet{bansak_2018} propose a machine learning-based algorithm for family placement that aims to optimize the overall employment rate of refugees. However, this method ignores families’ preferences.

\citet{olbergml} propose two matching mechanisms that, additionally to the optimization of the employment success, take into account also refugees’ preferences over locations. These innovative methods improves average resettlement success, allowing a trade off between family welfare and overall employment success.

According also to the formally notation of \citet{salles}, an instance of a refugee-country matching problem is a 6-tuple \((C, R, q, P_c, P_r, F)\), where \(C = \{c_1, \dots, c_m\}\) and \(R = \{r_1, \dots, r_n\}\) are disjoint sets of \(m\) countries and \(n\) refugees, respectively.

Since the distribution for refugee resettlement is modelled as a matching problem under preferences, refugees and countries offering asylum can be considered as a two-sided market in which the members of one side are distributed over the members of the other side. So, we can define the agents of the market as \(a_k \in C \cup R\).

Note that we are concerned with \textit{many-to-one matchings} since it can be assumed that \(n \gg m\) and each refugee can obtain asylum in at most one country, whereas a given country can accept many refugees. The maximum number of refugees that can be matched to each country is determined by a vector of quotas \(q = (q_j)_j \in \mathbb{N}^m,\ j \in \{1, \dots, m\}\). There may be no real quotas at all: setting \(q_j = n\ \ \forall c_j \in C\) makes them dummies.

Afterwards, \(P_c = \{P(c_1), \dots, P(c_m)\}\) and \(P_r =\{P(r_1), \dots, P(r_n)\}\) are sets of preference lists which include a complete and transitive preference profile for each country over the set of refugees and for each refugee over the set of countries. Each preference \(P_r(r_i)\) contains a list of expressed preferences in the format \(c_1 \succ_{r_i} c_2\), and equivalently for countries' preferences (an example is illustrated in Table~\ref{tab:countries-refugees}).

In the particular case of refugees' allocation, last elements to consider are the groups of people who must be assigned together (e.g. a family): let \(F=\{F(f_1), \dots, F(f_l)\}\) be the list of the groups of refugees, where \(F(f_i) = \{r_a, r_b, \dots\}\) and \(l \leq n\).
%In this case you should be careful to do not exceed the maximum bound to the number of people to keep in the same group and be careful to remove duplicate entries of the same people.
An example of a full matching is shown in Figure~\ref{fig:complete_matching}.

A country may declare some refugees unacceptable and a refugee may declare some countries unacceptable, hence, \(E \subseteq R \times C \times F\) is the subset of the acceptable refugee-country pairs. In addition, the amount of acceptable tuples is bounded because, as a refugee can't be split into two nations, nor can one nation receive more than the maximum allowable quotas. Denote \(A \left( r_i \right) = \left\{ c_j \mid \left( r_i, c_j \right) \in E \right\}\) as the set of acceptable countries for a given \(r_i \in R\); and equivalently for the countries.

An assignment \(M\) is a subset of \(E\) that contains \(a_k \in R \cup C\) items. Obviously a refugee \(r_i\) can be unassigned: \(M \left( r_i \right) = \emptyset\). Similarly, a country \(c_j\) can admit asylum requests if \(\left| M \left( c_j \right) \right| < q_j\) and, therefore, requests are blocked if \(\left| M \left( c_j \right) \right| = q_j\). Note that the assignment is valid if and only if \(\left| M \left( r_i \right) \right| \leq 1\ \ \forall r_i \in R\) and \(\left| M \left( c_j \right) \right| \leq q_j\ \ \forall c_j \in C\).

\begin{table}[!htb]
    \centering
    \begin{tabular}{c|c}
        \hline Countries & Refugees \\
        \hline \(r_1 \succ_{c_1} r_2 \succ_{c_1} r_3\) & \(c_2 \succ c_1\) for both \(r_1\) and \(r_2\) \\
        \(r_2 \succ_{c_2} r_1 \succ_{c_2} r_3\) & \(r_3\) declares only \(c_2\) acceptable \\
        \hline
    \end{tabular}
    \caption{Refugees and countries' preferences example. Refugees \(r_1, r_2, r_3\); countries \(c_1, c_2\) with \(q_1 = 2,\ q_2 = 1\). Notation \(a \succ_c b\) denotes that \(c\) strictly prefers \(a\) to \(b\).}
    \label{tab:countries-refugees}
\end{table}

\begin{figure}[!htb]
    \def\svgwidth{\columnwidth}
    \subfile{media/complete_matching.pdf_tex}
    \caption{Full matching example. Families \(f_1, \dots, f_5\) with \(|f_1|=|f_4|=|f_5|=3,\ |f_2|=1,\ |f_3|=2\); countries \(c_1, c_2\) with \(q_1 = 6,\ q_2 = 7\).}
    \label{fig:complete_matching}
\end{figure}
